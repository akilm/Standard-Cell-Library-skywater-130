\documentclass[conference]{IEEEtran}
\IEEEoverridecommandlockouts
% The preceding line is only needed to identify funding in the first footnote. If that is unneeded, please comment it out.
\usepackage{cite}
\usepackage{amsmath,amssymb,amsfonts}
\usepackage{algorithmic}
\usepackage{graphicx}
\usepackage{textcomp}
\usepackage{xcolor}

\usepackage[hidelinks]{hyperref}
\usepackage{url}
\def\BibTeX{{\rm B\kern-.05em{\sc i\kern-.025em b}\kern-.08em
    T\kern-.1667em\lower.7ex\hbox{E}\kern-.125emX}}
\begin{document}

\title{ Standard Cell Library Design, Layout and Characterization}

\author{\IEEEauthorblockN{Akil M}
\IEEEauthorblockA{\textit{Electrical and Electronics Engineering, NIT-Trichy}
}
}

\maketitle

\begin{abstract} This Report Describes the Design flow and Characterisation steps involved in the design of Standard Library cells using the Skywater 130 nm PDK. Standard cell Library plays a crucial role in the physical design flow or the backend flow of modern ASIC design. These cells are used by the synthesizer to convert the design specified in any HDL to a circuit level netlist of different blocks.  
\end{abstract}

\section{Introduction}
Standard Cell Library is an assortment of low-level logic cells which can implement basic Combinational and sequential logic functions. They can be categorized as a) Combinational Logic Cells - NAND, NOR, INV, BUF b) Sequential Cells - D Flip-Flops and Latches c) Special cells - Filler cells, Decap cells, Tap cells and Antennas. The standard cell library contains both the layout (physical implementation) of the cells and other files that describe the characteristics of the cell. 
\section{Design Flow}
The design of Standard cell is divided into different stages as discussed in this section.
\subsection{Pre-Layout Simulation}
The first step in the design of standard cells is the Pre-Layout simulation of the logic block. SPICE simulation tools (NgSpice) are used to verify the logical functionality of the cells and determine the widths and lengths of each transistor used in the logic block. The input and output waveforms are analyzed and the widths are adjusted to approximately match the required specifications. 
\subsection{Layout Design}
The layout is drawn using open-source layout tools like MAGIC. Using Euler's Path, the inputs are ordered in such a way that the layout results in minimum area and less complex. All the standard cells are to be designed as fixed-height and variable-width cells. DRC (Design Rules Check) and LVS (Layout versus Schematic) checks are done to ensure that there is no violations in the designed layout.
\subsection{Post-Layout Simulation}
Various Parasitic Capacitances which are present in the layout are extracted using the layout tool as a SPICE file. Post-Layout simulation is done on the extracted SPICE file to ensure that the layout matches the specifications required. Parasitics play a huge role in altering the performance of the standard cell. The differences betweeen the simulation results obtained in the Pre-layout and Post-layout stage is mainly due to the presence of these Parasitic Capacitances. 
\section{Characterization}
Characterization involves characterizing the different timing specifications of the circuit under different output loads and input skew. This step involves finding the rise transition time, fall transition time,	cell rise delay, cell fall delay ,etc. The Standard cell is characterised by simulating the netlist extracted post layout using SPICE. Ngspice commands can be used to save the output and input waveforms as a .raw file. These .raw files contains the vectors with each vector representing the voltage values of input, output and other nodes along with a time-stamp. The .raw files can be processed using python scripts to automatically find the different parameters like rise and fall times etc. Alternatively, Pyspice can be used to characterise the results, PySpice implements a Ngspice binding and provides an oriented object API on top of SPICE, the simulation output is converted to Numpy arrays for convenience which can be processed easily .After obtaining the values of the different parameters, they can be ordered according to the liberty file format (.lib) which are basically text files which are structured in an predefined order and contain all the values that characterise the standard cell. 
\begin{flushleft}
\begin{figure}[!htb]
\includegraphics[scale=0.35]{sdfs.png}
\caption{Steps for Standard Cell Design,Layout and Characterisation }
\end{figure}
\end{flushleft}
\section{Conclusion and Future Scope}
There are not many open source tools for the characterisation of standard cells. Automating the characterisation of standard cells using python scripts could setup a base for developing a quality tool for timing characterisation of standard cells. The circuit topology can also be improved by researching newer topologies for optimising the delay, area and power.

\begin{thebibliography}{00}

\bibitem{b1}  Naga Lavanya, Pradeep Mullangi. "Design and Development of an ASIC Standard Cell Library Using 90nm Technology Node". 2018 International Conference on Computer Communication and Informatics (ICCCI -2018).  
\bibitem{b2} Srujan R, Sruja m, Vinay S, Vishal M S. "Design, Implementation and Characterization of 45nm Standard Cell Library for Industrial Synthesis Flow" IJERT; NCESC - 2018 Conference Proceedings
\end{thebibliography}

\end{document}
